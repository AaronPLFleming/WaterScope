\documentclass[12pt]{article}
\usepackage{amsmath}
\usepackage{amsfonts}
\usepackage{pifont}
\newcommand{\tick}{\ding{51}}
\newcommand{\cross}{\ding{55}}
\usepackage[dvipsnames]{xcolor}
\newcommand{\e}[1]{{\mathbb E}\left[ #1 \right]}
\textwidth 18cm \oddsidemargin -0.5cm \topmargin -2.25cm \textheight
25cm \footskip 0.7cm \usepackage{epsfig}
\usepackage{amsmath,graphicx,psfrag,pstcol}
\def\n{\noindent}
\def\u{\underline}
\def\hs{\hspace}
\newcommand{\thrfor}{.^{\displaystyle .} .}
\newcommand{\bvec}[1]{{\bf #1}}
\graphicspath{{./Figures/}}

\begin{document}

\noindent
\rule{15.7cm}{0.5mm}


\begin{center}
{\bf Technology for the Poorest Billion}\\
\vspace{0.5cm} {\bf Waterscope - Project Overview}\\
\end{center}
\rule{15.7cm}{0.5mm}

\hfill\\
\section{Proposal}
\subsection{Overview, Context and Approach}
\subsubsection{Waterscope}
Waterscope is a small company formed in 2015 to attempt to combat the lack of access to safe drinking water. Currently, waterborne diseases from bacteria results in over 2.2 million deaths per year - the aim of Waterscope is to devise a simple, fast, water testing kit which can be used to quantitatively test water sources on site for bacterial levels within a few hours.
\subsubsection{The Problem}
As part of this testing kit, water is filtered through an extremely fine filter, which collects (some of) the bacteria. However, the current system (using a syringe) is slow and energetic, and therefore unsuitable for widespread use by people of a variety of strengths and abilities etc. The task of our team is to design a mechanism for filtering that allows a more rapid, and easier, method of passing the required volume of water ($\approx 100$ml) through the filter.\\
A second issue regarding this system is cost. While the overall testing apparatus can be of a reasonable cost (several $\pounds$100), the components used for filtering (which, since they require sterility, are designed to be disposable) must be very cheap. This makes the current solution of a syringe impractical, since even the relatively low cost of a syringe ($\approx$ 50p) is prohibitive.
\subsubsection{Approaches}
Two main approaches occurred to us. The first option was to consider a different syringe-like device which allows the application of a greater force, thereby increasing flow rate. However, we reasoned that this would be even more expensive than the original solution - designing our own custom syringe is unlikely to be of a practical cost.\\
The second idea, and the idea which the WaterScope team recommended that we pursue, was to design a vacuum filtering system. We decided that it would be relatively feasible to design a cheap vacuum pump that is cheap and easy to manufacture, ideally in-country, as well as a sealed interface between filter and collection vessel. We plan to use a glass flask as our collection vessel for now, in the understanding that modifications could be made relatively easily in the future by the WaterScope team if a glass flask proves to be impractical. The benefit of this solution is, in theory, its low cost - since the pump and flask both lie downstream of the filter (as well as large portions of our proposed interface system), they are not required to be disposable and therefore are permitted to be more expensive.
\subsection{Technical Aspects}

\subsection{Project Management Plan}

\subsubsection{Timeline?}

\subsubsection{Team Member Strengths etc.}
Aaron - dashing good looks

\subsubsection{Skills/Training Required}

\subsection{Costing \& Parts}

\subsection{Risk Assessment}

\subsection{Contingency Plan}



\section{Interim}

\section{Presentation}

\section{Final Report}

\end{document}
