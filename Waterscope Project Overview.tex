\documentclass[12pt]{article}
\usepackage{amsmath}
\usepackage{amsfonts}
\usepackage{pifont}
\newcommand{\tick}{\ding{51}}
\newcommand{\cross}{\ding{55}}
\usepackage[dvipsnames]{xcolor}
\newcommand{\e}[1]{{\mathbb E}\left[ #1 \right]}
\textwidth 18cm \oddsidemargin -0.5cm \topmargin -2.25cm \textheight
25cm \footskip 0.7cm \usepackage{epsfig}
\usepackage{amsmath,graphicx,psfrag,pstcol}
\def\n{\noindent}
\def\u{\underline}
\def\hs{\hspace}
\newcommand{\thrfor}{.^{\displaystyle .} .}
\newcommand{\bvec}[1]{{\bf #1}}
\graphicspath{{./Figures/}}

\begin{document}

\noindent
\rule{15.7cm}{0.5mm}


\begin{center}
{\bf Technology for the Poorest Billion}\\
\vspace{0.5cm} {\bf Waterscope - Project Overview}\\
\end{center}
\rule{15.7cm}{0.5mm}

\hfill\\
\section{Proposal}
\subsection{Overview, Context and Approach}
\subsubsection{Waterscope}
Waterscope is a small company formed in 2015 to attempt to combat the lack of access to safe drinking water. Currently, waterborne diseases from bacteria results in over 2.2 million deaths per year - the aim of Waterscope is to devise a simple, fast, water testing kit which can be used to quantitatively test water sources on site for bacterial levels within a few hours.
\subsubsection{The Problem}
As part of this testing kit, water is filtered through an extremely fine filter, which collects (some of) the bacteria. However, the current system (using a syringe) is slow and energetic, and therefore unsuitable for widespread use by people of a variety of strengths and abilities etc. The task of our team is to design a mechanism for filtering that allows a more rapid, and easier, method of passing the required volume of water ($\approx 100$ml) through the filter.\\
A second issue regarding this system is cost. While the overall testing apparatus can be of a reasonable cost (several $\pounds$100), the components used for filtering (which, since they require sterility, are designed to be disposable) must be very cheap. This makes the current solution of a syringe impractical, since even the relatively low cost of a syringe ($\approx$ 50p) is prohibitive.
\subsubsection{Approaches}
Two main approaches occurred to us. The first option was to consider a different syringe-like device which allows the application of a greater force, thereby increasing flow rate. However, we reasoned that this would be even more expensive than the original solution - designing our own custom syringe is unlikely to be of a practical cost.\\
The second idea, and the idea which the WaterScope team recommended that we pursue, was to design a vacuum filtering system. We decided that it would be relatively feasible to design a cheap vacuum pump that is cheap and easy to manufacture, ideally in-country, as well as a sealed interface between filter and collection vessel. We plan to use a glass flask as our collection vessel for now, in the understanding that modifications could be made relatively easily in the future by the WaterScope team if a glass flask proves to be impractical. The benefit of this solution is, in theory, its low cost - since the pump and flask both lie downstream of the filter (as well as large portions of our proposed interface system), they are not required to be disposable and therefore are permitted to be more expensive.
\subsection{Technical Aspects}

\subsection{Project Management Plan}
Project Management Plan

Our project outline will be centred on the following Gantt chart, which details the specific tasks to be completed by certain dates. 

We are using Slack for all project communication, both between team members and to communicate with the team at Waterscope. 

Github is being used for all file uploads, to track progress and changes made by individual group members, and to show who has contributed to which parts of the group projects.

We plan to complete the first project proposal collaboratively, by working on separate parts, uploading them to GIT and then adding and editing the work of our peers as we best see fit.


The following Gantt Chart, team strengths and training plans detail the key points of the project management plan to be discussed.
\subsubsection{Timeline?}

\subsubsection{Team Member Strengths etc.}
Aaron - dashing good looks

\subsubsection{Skills/Training Required}

During the course of the 4 week project, the whole team will be required to draw up upon both technical and team working skills. However, we recognise that to produce the best possible outcome we will benefit from certain members also learning a number of new skills for specialised tasks. Examples of beneficial skills are as follows:  

\begin{itemize}
\item CAD/modelling: Computer aided design is the most efficient method for creating any models of proposed designs, due to it's easy to refine/edit nature. We anticipate initial design ideas being expressed via hand drawings but the skill of translating the best contenders into CAD models quickly and accurately will likely prove valuable when determining the performance and manufacturing feasibility of our design ideas. 
\item 3D printing/rapid-protyping: The ability to create physical models of our designs is going to be essential in not only testing but realising our designs. The 3D printers available to in the department are a fantastic tool for creating complex geometries quickly and over the course of the project most members of our group will learn/utilise the skill of turning CAD models into products using 3D printing. 
\item Laser/plasma cutter: We also anticipate the need to use other manufacturing methods when making our prototypes, and will endeavor to learn the skills to use the Dyson Centre's cutting tools if the need arises. 
\item Web Design: Our final proposal will be presented in the form of a website that provides a detailed but intuitive overview of the teams progress towards a solution. This will require coding skills in languages such as HTML and Javascript, and graphic/user interface design skills to make sure the website functions ideally and is easy for the end user to navigate.  
\end{itemize}
 
\n We also anticipate our proficiency when functioning as a team will depend greatly on certain 'soft skills' such as: 

\begin{itemize}
\item Time Management - Gantt Chart 
\item Communication Skills - slack 
\item Workload sharing/delegation - Github 
\end{itemize}

\subsection{Costing \& Parts}

\subsection{Risk Assessment}

\begin{table}[h]
\centering
\begin{tabular}{|l|l|l|l|}
\hline
Hazard & Effect & How to control risk & Risk level \\ \hline
Glass flask & \begin{tabular}[c]{@{}l@{}}Breakage of flask can \\ cause injury with \\ sharp glass edges\end{tabular} & \begin{tabular}[c]{@{}l@{}}Taking care when \\ handling flask\end{tabular} & Medium \\ \hline
Handling plastic bags & Risk of suffocation from bag & \begin{tabular}[c]{@{}l@{}}Don't hold bag \\ over nose or mouth\end{tabular} & Low \\ \hline
\begin{tabular}[c]{@{}l@{}}Working in the \\ Dyson Centre \end{tabular} & \begin{tabular}[c]{@{}l@{}}Various injuries if \\ using tools improperly\end{tabular} & \begin{tabular}[c]{@{}l@{}}All team members must be \\ properly trained\end{tabular} & Medium \\ \hline
Laser/Plasma cutting & Possible injuries to extremities & Proper training & Medium \\ \hline
\begin{tabular}[c]{@{}l@{}}Extended use of \\ computer screens \end{tabular} & Straining of eyes & Limit computer usage & Low \\ \hline
\end{tabular}
\end{table}

\subsection{Contingency Plan}

What could go wrong, and what can we do about it?\\
\textit{Would prefer to flesh this out once we know more about the specifics of the project}
\begin{itemize}
\item
\end{itemize}


\section{Interim}

\section{Presentation}

\section{Final Report}

\end{document}
